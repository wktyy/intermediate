\documentclass[twocolumn]{jsarticle}
\usepackage[dvipdfmx]{graphicx}
\usepackage{booktabs}
\usepackage{mathcomp}
\usepackage{array}
\usepackage{mathtools,amssymb}
\usepackage{siunitx}
\usepackage{multirow}
\usepackage{tabularx}
\usepackage{subcaption}
\usepackage{float}
\usepackage{listings,jvlisting}
\lstset{
	basicstyle={\ttfamily},
	identifierstyle={\small},
	commentstyle={\smallitshape},
	keywordstyle={\small\bfseries},
	ndkeywordstyle={\small},
	stringstyle={\small\ttfamily},
	frame={tb},
	breaklines=true,
	columns=[l]{fullflexible},
	numbers=left,
	xrightmargin=0zw,
	xleftmargin=3zw,
	numberstyle={\scriptsize},
	stepnumber=1,
	numbersep=1zw,
	lineskip=-0.5ex,
	tabsize=2
}
\renewcommand{\lstlistingname}{ソースコード}
\title{仮想筋電義手の開発に関する研究}
\author{河合 将暉  指導教官 戸崎 哲也}
\date{}

\begin{document}
\maketitle

\section{はじめに}
	
	\section{研究内容}
		\subsection{3Dスキャナ}
		\subsection{blender}
		\subsubsection{ボーン・}
		\subsection{unity}
		

\section{研究結果}
	\subsection{シミュレータの構成}
\section{まとめ}
	今後の課題としては, 腕の動きをVR上に変換するためのインタフェースをマウスによって構成しているが, 

\begin{thebibliography}{99}
\bibitem{ref:}
\end{thebibliography}
\end{document}