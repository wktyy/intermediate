\documentclass[11pt]{ltjsarticle}
%\usepackage[dvipdfmx]{graphicx}
\usepackage{graphicx}
\usepackage{booktabs}
\usepackage{mathcomp}
\usepackage{array}
\usepackage{mathtools,amssymb}
\usepackage{siunitx}
\usepackage{multirow}
\usepackage{tabularx}
\usepackage{subcaption}
\usepackage{float}
\usepackage{abstract}

\title{仮想筋電義手の開発に関する研究}
\author{河合将暉}
\adviser{戸崎哲也}

\thispagestyle{empty}
\pagestyle{empty}

\begin{document}
\maketitle

\section{はじめに}
	
	\section{研究内容}
		\subsection{3Dスキャナ}
		\subsection{blender}
		\subsubsection{ボーン・}
		\subsection{unity}
		有料か

\section{研究結果}
	\subsection{シミュレータの構成}
\section{まとめ}
	今後の課題としては, 腕の動きをVR上に変換するための
	インタフェースをマウスによって構成しているが,

\begin{thebibliography}{99}%参考文献
	\bibitem{ref:1}
	Taro Denshi : ``How to write'', Jpn. J. KCCT,
	\textbf{6}, pp.100-200 (2001).
	\bibitem{ref:2}
	高専 太郎 : ``論文記述法'', 神戸出版, pp.51-200 (2001).
\end{thebibliography}
\end{document}