\documentclass[report]{ltjsbook}
%\usepackage[dvipdfmx]{color}
\usepackage{booktabs}
\usepackage{mathcomp}
\usepackage{array}
\usepackage{mathtools,amssymb}
\usepackage{siunitx}
\usepackage{multirow}
\usepackage{tabularx}
\usepackage{subcaption}
\usepackage{float}
\usepackage{listings,jvlisting}
\lstset{
	basicstyle={\ttfamily},
	identifierstyle={\small},
	commentstyle={\smallitshape},
	keywordstyle={\small\bfseries},
	ndkeywordstyle={\small},
	stringstyle={\small\ttfamily},
	frame={tb},
	breaklines=true,
	columns=[l]{fullflexible},
	numbers=left,
	xrightmargin=0zw,
	xleftmargin=3zw,
	numberstyle={\scriptsize},
	stepnumber=1,
	numbersep=1zw,
	lineskip=-0.5ex,
	tabsize=2
}
\renewcommand{\lstlistingname}{ソースコード}



\title{仮想筋電義手の開発に関する研究}
\author{神戸市立工業高等専門学校 \\ 電子工学科 河合 将暉}


\begin{document}
\maketitle

\setcounter{tocdepth}{3}
\tableofcontents
\clearpage

\chapter{はじめに}

\chapter{解説}
	\section{EinScan HX}
		EinScan HXは株式会社サンステラが提供するハンディ3Dスキャナである.
		主な使用用途としては,工業製品などの比較的大きな物をスキャンし,
		リバースエンジニアリングや測定などに用いられている.
		製品仕様については次項で解説する.
		\subsection{製品仕様}
			EinScan HXの製品仕様について表\refeq{tab:EinScan}に示す.
			\begin{table}[H]
			\begin{center}
			\caption{EinScan HXの製品仕様}
			\label{tab:EinScan}
			\begin{tabular}{c|cc} \toprule
				スキャンモード&Rapidスキャン&レーザースキャン\\ \hline
				スキャン精度&0.05\,mm&0.04\,mm\\
				ポイント間隔&0.25~3.00\,mm&0.05~3.00\,mm\\
				被写体長3D精度&±0.1\,mm&±0.06\,mm\\
				シングルスキャン精度&$420 \time 440$\,mm&\\
				光源&ブルーLED&ブルーレーザー7本\\
				被写体深度&&\\
				対象物との距離&&\\
				テクスチャスキャン&&\\
				安全性&&\\
				データ出力&&\\
				本体サイズ&&\\
				本体重量&&\\
				\bottomrule
			\end{tabular}
			\end{center}
			\end{table}
			ここで,被写体長3D精度というのはスキャン時に取得するポイントの
			最大累積誤差を示したもので,より大きな物体をスキャンする際に
			誤差として現れるものである.
	\section{Blender}
		\subsection{スムージング}
		\subsection{ボーン構成}
		\subsection{ウェイトペイント}
	\section{Unity}
		\subsection{座標系}
		\subsection{シェーダー}
			\subsubsection{既存シェーダー}
			\subsubsection{Reflex Shader 2.2}
		\subsection{カメラ}
		\subsection{C\#スクリプト}
		\subsection{オブジェクト}
		\subsection{衝突判定}
		\subsection{ペアレント}
		\subsection{コンストレイント}
\clearpage
		\subsection{物理演算}
		\subsection{エディタ設定}
		\subsection{プラグイン}
			\subsubsection{FVRsdk}
			\subsubsection{Xcharts}
			\subsubsection{Android Logcat}
			\subsubsection{XR Interaction Toolkit}
			\subsubsection{Google VR}
		\subsection{ビルド}
			\subsubsection{iOS}
			\subsubsection{Xcode}
			\subsubsection{Android OS}
	\section{FirstVR}
		\subsection{デバイス構成}
		\subsection{筋変位センサ}
		\subsection{トラッキング}
		\subsection{キャリブレーション}
		\subsection{BLE通信}
	\section{Meta Quest2}
		\subsection{デバイス構成}
	\section{rasberry py}
	
\chapter{研究内容}
	\section{使用器具}
	\section{3Dスキャナ}
		\subsection{3Dモデルの取り込み}
		\subsection{出力形式の選定}
	\section{Blender}
		\subsection{スムージング処理}
		\subsection{ボーン配置}
	\section{Unity}
		\subsection{シェーダー選定}
		\subsection{}
\chapter{研究結果}
\chapter{まとめ}
\chapter{今後の課題}
\begin{thebibliography}{99}

	\bibitem{ref:1}
	芝軒 太郎 他.``VRを利用した筋電義手操作トレーニング
	システムの開発と仮想 Box and Block Test の実現''.
	JRSJ. 2012 July.

	\bibitem{ref:2}
	Osumi M, et al.
	``Characteristics of Phantom Limb Pain Alleviated
	with Virtual Reality Rehabilitation''.
	Pain Med. 2019 May.

	\bibitem{ref:3}
	H2L.Inc.,Tokyo106-0032,Japan;satoshi.hosono@h2l.jp

	\bibitem{ref:4}
	Tamon Miyake, etal``Gait Phase Detection Based on Muscle Deformation
	with Static Standing-Based Calibration''.
	MDPI. 2021 Feb

	\bibitem{ref:5}
	株式会社サンステラ,https://www.einscan.jp/einscan-hx

\end{thebibliography}
\end{document}