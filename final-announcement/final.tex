\documentclass{ltjsarticle}
%\usepackage[dvipdfmx]{graphicx}
\usepackage{graphicx}
\usepackage{booktabs}
\usepackage{mathcomp}
\usepackage{array}
\usepackage{mathtools,amssymb}
\usepackage{siunitx}
\usepackage{multirow}
\usepackage{tabularx}
\usepackage{subcaption}
\usepackage{float}
\usepackage{setspace}
\usepackage{abstract}

\title{仮想筋電義手の開発に関する研究}
\author{河合 将暉}
\adviser{戸崎 哲也}

\thispagestyle{empty}
\pagestyle{empty}

\begin{document}
\maketitle

\section{はじめに}
	中間報告までに構成したVRトレーニングシステムのインタフェースを
	実際の上肢切断者にも使用可能にするため,筋変位センサであるFirstVR
	を用いてVRトレーニングシステムの再構成およびその評価を行った.
\section{解説}
	\subsection{FirstVR}
		FirstVRとは,H2L株式会社が提供する筋変位VRコントローラである.
		コントローラに搭載されているセンサとしては,3軸ジャイロセンサ,
		3軸加速度センサ,3軸磁気センサ,14チャンネルの光学筋変位センサ
		が搭載されている.対応しているOSはiOS/Android OSに対応しており,
		推奨スペックがiphone 7/GALAXY S8 となっている.OSとの通信は
		BLE通信で行われており,Bluetooth®4.2が搭載されている.

	\subsection{ジェスチャ認識}
		FirstVRではジェスチャを認識することができ,その方式として,
		特定のジェスチャをしている状態の筋変位値を閾値とすることで,
		何もしていない状態とジェスチャを行っている状態を区別して
		認識している.
	\subsection{BLE通信}
		BLE通信とは,Bluetooth Low Energyの略称であり,
	\subsection{iOS/Android OSのビルド}

\section{研究内容}
	\subsection{FirstVRの接続確認}
		まず,FirstVRとUbuntu22.04との接続を試みたが,接続できなかったため,
		公式が提供しているiOSアプリケーションからFirstVRが初期動作不良ではないか
		を確認した.確認後,アプリケーション上での通信・動作確認が行えたため,
		Unity上でサンプルシーンをiOS用にビルドし,ビルドしたファイルをMacOSに
		送信後,Xcodeでiphone13(iOS16.7.2)にアプリケーションを作成し,動作・接続確認をした.

	\subsection{iOS/Android OSでのビルド}
		
	\subsection{サンプルシーンの検証}
	\subsubsection{筋変位センサの測定}
	\subsubsection{ジェスチャ認識機能の検証}
	\subsubsection{キャリブレーション機能の検証}
	\subsection{VRトレーニングシステムへの置換}
	\subsection{iOS/AndroidOSでのシステム構成}
	\subsection{Meta Quest2でのシステム構成}
	\subsection{評価方法}

\section{研究結果}
\section{まとめと今後の課題}

\begin{thebibliography}{99}%参考文献
	\begin{spacing}{0.8}

		\bibitem{ref:1}
			H2L.Inc.,Tokyo106-0032,Japan;satoshi.hosono@h2l.jp
		\bibitem{ref:2}
			Tamon Miyake, etal``Gait Phase Detection Based on Muscle Deformation
			with Static Standing-Based Calibration''.
			MDPI. 2021 Feb

	\end{spacing}
\end{thebibliography}
\end{document}