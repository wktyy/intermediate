\documentclass{ltjsarticle}
%\usepackage[dvipdfmx]{graphicx}
\usepackage{graphicx}
\usepackage{booktabs}
\usepackage{mathcomp}
\usepackage{array}
\usepackage{mathtools,amssymb}
\usepackage{siunitx}
\usepackage{multirow}
\usepackage{tabularx}
\usepackage{subcaption}
\usepackage{float}
\usepackage{setspace}
\usepackage{abstract}

\title{仮想筋電義手の開発に関する研究}
\author{河合 将暉}
\adviser{戸崎 哲也}

\thispagestyle{empty}
\pagestyle{empty}

\begin{document}
\maketitle

\section{はじめに}
	上肢切断者が筋電義手を装着する際,自在に扱うことができるように
	訓練を行う必要があり,VRシミュレータを用いた訓練効果については
	先行研究\cite{ref:1}で検討されている.本研究では,仮想筋電義手
	3Dモデル(VH:Virtual Hand)のリアリティによる訓練効果に着目し3D
	スキャナで取り込んだVHを用いたVRシミュレータを構成し,
	その見た目によって訓練効果に変化があるかを評価することを目的とする.
	前段階として,インタフェースの性能評価およびVRシミュレータの構成と評価を行った.
\section{解説}
	\subsection{FirstVR}
		FirstVR\cite{ref:2}とは,\,H2L株式会社が提供する筋変位VRコントローラである.
		コントローラに搭載されているセンサとしては,\,3軸ジャイロセンサ,\,
		3軸加速度センサ,\,3軸磁気センサ,\,14チャンネルの光学筋変位センサ
		が搭載されている.対応しているOSはiOS/Android OSに対応しており,\,
		OSとの通信はBLE通信で行われている.

	\subsection{ジェスチャ認識}
		FirstVRでは光学筋変位センサの測定値を用いてジェスチャを認識することができる.
		その方式として,特定のジェスチャをしている状態の筋変位値を閾値とすることで,
		何もしていない状態とジェスチャを行っている状態を区別して認識することが可能である.

\section{研究内容}
	\subsection{FirstVRの性能評価手法}
		本シミュレータのインタフェースであるFirstVRにおいてジェスチャ認識の精度を確認するため,
		実験協力者として電子工学科5年生の学生31名(男性26名:女性5名)を対象にFirstVRの評価を行う.
		まず,じゃんけんのグーのジェスチャを学習させ,ジェスチャの検出精度を測定した.
		加えて,評価指標として総変化量$X$を定め,各sample数ごとに分散を調べ,最適なsample数の検討を行う.
		総変化量の算出は測定回数$s = 5$,チャンネル数$r = 14$としてジェスチャ状態で測定した筋変位センサの値を$M_{{s}{r}}$
		とジェスチャしていない状態の筋変位センサの値$N_{r}$とすると式\refeq{eq:originaldata}と示すことができる.
		\begin{equation}
			\label{eq:originaldata}
			X = \frac{1}{5} \sum_{s = 1}^{5} \sum_{r = 0}^{13} |N_{r} - M_{{s}{r}}|
		\end{equation}
		

	\subsection{シミュレータの構成}
		Blenderで処理したVHをUnityにインポートし,入力インターフェースとしてキーボード・マウス
		を用いるPC版と,\,FirstVRを用いるiOS版の2種類を構成した.
		
	\subsection{シミュレータの定性評価手法}
		電子工学科5年生の男性11名に協力していただき,\,PC版シミュレータとiOS版シミュレータの2種類の操作説明を行う.
		その後,シミュレータを各5分程度体験してもらい,それぞれの没入感,操作性,応答性の3項目について6段階リッカート尺度を用いた定性評価
		アンケート調査を行う.
		評価点数が高いほどそれぞれの項目において高得点の評価となるように設定し,調査結果に対して分析を行う.

\section{研究結果}
		結果として,\,sample数に依存しない誤検知のデータを除外するとジェスチャ検知精度はどのsample数においても96.6 ± 3.34\,\%のジェスチャ認識率が結果として得られた.
		また,総変化量の標本分散はsample70が最も小さく,次いでsample7,sample100の順に分散が少なくなることが得られた.
		ここで,\,sample数7,70,100以外のデータは分散がこの3種類よりも比較的大きく,外れ値も含んでいるため,安定して動作していると考えにくい.
		最終的に,この3種類の中で最もシミュレータに対する負荷が小さいデータとしてsample7を選定した.
		FirstVRの性能評価用アプリケーションの構成を図\refeq{fig:FirstVRapplication}に示す.
		\begin{figure}[H]
		\centering
		\begin{minipage}{0.9\columnwidth}
		\centering
		\includegraphics[width = \columnwidth]{../figs/IMG_1866.PNG}
		\subcaption{ジェスチャ未判定時}
		\label{fig:FVRnocalibration}
		\end{minipage}
		\hspace{0.04\columnwidth}
		\begin{minipage}{0.9\columnwidth}
		\centering
		\includegraphics[width = \columnwidth]{../figs/IMG_1867.PNG}
		\subcaption{ジェスチャ判定時}
		\end{minipage}
		\caption{FirstVR性能評価用アプリケーションの構成}
		\label{fig:FirstVRapplication}
		\end{figure}
		\vspace{-15pt}


	\subsection{シミュレータの構成}
		各シミュレータの画面構成を図\refeq{fig:simurate}に示す.
		\begin{figure}[H]
		\centering
		\begin{minipage}{0.9\columnwidth}
		\centering
		\includegraphics[width = \columnwidth]{../figs/PCnomal.png}
		\subcaption{PC版の画面構成}
		\end{minipage}
		\hspace{0.04\columnwidth}
		\begin{minipage}{0.9\columnwidth}
		\centering
		\includegraphics[width = \columnwidth]{../figs/iOSnomal.png}
		\subcaption{iOS版の画面構成}
		\end{minipage}
		\caption{シミュレータの実行画面}
		\label{fig:simurate}
		\end{figure}
		\vspace{-25pt}

	\subsection{シミュレータの定性評価}

		PC版ではキーボード・マウスを入力インターフェースとしているため,
		学習などの準備は必要なく,基本的な操作説明のあとにシミュレータを評価した.
		iOS版ではiPhoneにシミュレータを表示させ,\,FirstVRのジェスチャ認識機能
		によってじゃんけんのグーの状態を学習することでオブジェクト保持ができる.
		この学習が終了してから約5分間シミュレータを評価した.アンケート結果を
		表\refeq{tab:resoult}に示す.
		
		\vspace{-10pt}
		\begin{table}[H]
		\begin{center}
		\caption{アンケート結果の統計}
		\label{tab:resoult}
		\begin{tabular}{cc|ccc} \Hline
		&評価項目&平均点&中央値&分散\\ \hline
		&没入感&3.36&4&1.14\\
		PC&操作性&3.09&3&0.992\\
		&応答性&3.54&4&2.07\\ \hline
		&没入感&3.90&4&2.63\\
		iOS&操作性&3.45&3&1.88\\
		&応答性&2.82&2&2.33\\ \Hline
		\end{tabular}
		\end{center}
		\end{table}

		表\refeq{tab:resoult}より,没入感,操作性の点ではiOS版の方が平均点が高くなっている
		ことがわかる.しかし,分散は3項目全てにおいてPC版の方が少なくなっており,
		iOS版はかなり評価点数に個人差があると読み取ることができる.


\section{おわりに}
	本研究ではPC版のシミュレータとiOS版のシミュレータの2種類を作製し,健常者に
	体験してもらい,その定性評価を行うことでFirstVRを用いたシミュレータの方が没入感
	が高い傾向があることを示した.しかし,操作性の点では多少の優位性を得ることができたが,
	装着者各個人による評価点の分散が大きいことが課題となった.また,遅延に関してもPC版の方
	が遅延が少ないという課題点も見つかった.そのうえ,表示するモニターの違いで没入感が違った
	という意見も挙げられており,それについても検討する必要がある.

	今後の展望はシミュレータのVHモデルをスキャンしたモデルから長方形などのモデルに変更し
	比較することで見た目による訓練効果の違いについて検討していく予定である.


\begin{thebibliography}{99}%参考文献
	\begin{spacing}{0.8}

		\bibitem{ref:1}
			芝軒 太郎 他.``VRを利用した筋電義手操作トレーニング
			システムの開発と仮想 Box and Block Test の実現''. 
			JRSJ. 2012 July.

		\bibitem{ref:2}
			H2L.Inc.,Tokyo106-0032\\Japan;satoshi.hosono@h2l.jp

	\end{spacing}
\end{thebibliography}
\end{document}